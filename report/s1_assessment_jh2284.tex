\documentclass[12pt]{article}

\title{Principles of Data Science Coursework Report}
\author{James Hughes}

\usepackage{amsmath}
\DeclareMathOperator{\erf}{erf}

\begin{document}

\maketitle
\newpage


\section{Section A}

\subsection{Part (a)}

We begin by showing that both densities $s$ and $b$ are properly normalised in the range $M\in [-\infty, +\infty]$. In the former case, as a first step we use a change of variables $Z = \mu + \sigma M$ such that $\frac{dM}{dZ} = \sigma$, for which the integral limits don't change:

\begin{align*}
    \int_{-\infty}^\infty s(M;\mu, \sigma) & = \int_{-\infty}^\infty \frac{1}{\sqrt{2\pi}\sigma}\exp \left[-\frac{(M-\mu)^2}{2\sigma^2}\right]dM\\
        & = \int_{-\infty}^\infty \frac{1}{\sqrt{2\pi}\sigma}\exp(-\frac{1}{2}Z^2)\cdot \sigma dZ\\
        & = \frac{1}{\sqrt{2\pi}}\int_{-\infty}^\infty\exp(-\frac{1}{2}Z^2)dZ\\
\end{align*}

In order to prove that $s$ is properly normalised, we simply need to show that the last expression above evaluates to $1$. We do this by computing it's square, which in turn leads to an integral in two dummy variables. Below we then use the transformation to polar coordinates $(X,Y) = \rho(R,\phi) = (R\cos(\phi), R\sin(\phi))$ which has Jacobian matrix

\[
    \begin{pmatrix}
        \cos(\phi) & -R\sin(\phi) \\
        \sin(\phi) & R\cos(\phi) \\
    \end{pmatrix}
\]

and hence $|J_\rho(R,\phi)| = R$. Then, denoting the integral in the final expression above by $I$, we find:

\begin{align*}
    \left[\frac{1}{\sqrt{2\pi}}I\right]^2 & = \frac{1}{2\pi}\left[\int_{-\infty}^\infty\exp(-\frac{1}{2}X^2)dX\right]\left[\int_{-\infty}^\infty\exp(-\frac{1}{2}Y^2)dY\right] \\
        & = \frac{1}{2\pi}\int_{-\infty}^\infty\int_{-\infty}^\infty\exp(-\frac{1}{2}(X^2 + Y^2))dXdY\\
        & = \frac{1}{2\pi}\int_{0}^{2\pi}\int_{0}^\infty\exp(-\frac{1}{2}(R^2))\cdot R dRd\phi\\
        & = \left[-\exp(-\frac{1}{2}R^2)\right]_{R=0}^{R=\infty} \\
        & = 1. \\
\end{align*}

For the background, we note that the density (integrand) $b$ is zero for all $M<0$, and show
\begin{align*}
    \int_{-\infty}^\infty b(M;\lambda) & = \int_0^\infty \lambda e^{-\lambda M}dM \\
    & = \left[-e^{-\lambda M}\right]_{M=0}^{M=\infty} \\
    & = 1. \\
\end{align*}

Finally this lets us show that the probability density $p$ given is properly normalised over $[-\infty,+\infty]$ because
\begin{align*}
    \int_{-\infty}^\infty p(M; f,\lambda,\mu,\sigma)dM & = f\int_{-\infty}^\infty s(M;\mu, \sigma)dM + (1-f)\int_{-\infty}^\infty b(M;\lambda)dM \\
    & = f\cdot 1 + (1-f)\cdot 1\\
    & = 1. \\
\end{align*}

\subsection{Part (b)}

In order to ensure that the fraction of signal in the restricted distribution for $M$ remains $f$, we must normalise the signal and background separately over $[\alpha,\beta]$, we introduce these new restricted distributions as

\begin{align*}
    s_r(M;\mu,\sigma) = \frac{s(M;\mu,\sigma)}{\int_\alpha^\beta s(M;\mu,\sigma)dM} && b_r(M;\lambda) = \frac{b(M;\lambda)}{\int_\alpha^\beta b(M;\lambda)dM}
\end{align*}

We then compute the relevant integrals:

\begin{align*}
    \int_\alpha^\beta s(M;\mu,\sigma)dM & = \int_{-\infty}^\beta s(M;\mu,\sigma)dM - \int_{-\infty}^\alpha s(M;\mu,\sigma)dM \\
    & = \frac{1}{2}\left[1 + \erf\left(\frac{\beta - \mu}{\sigma}\right)\right] - \frac{1}{2}\left[1 + \erf\left(\frac{\alpha - \mu}{\sigma}\right)\right] \\
    & = \frac{1}{2}\erf\left(\frac{\beta - \mu}{\sigma}\right) - \frac{1}{2}\erf\left(\frac{\alpha - \mu}{\sigma}\right)
\end{align*}
\begin{align*}
    \int_\alpha^\beta b(M;\lambda)dM & = \int_{-\infty}^\beta b(M;\lambda)dM + \int_{-\infty}^\alpha b(M;\lambda)dM \\
    & = 1 - e^{-\lambda\beta} - (1 - e^{-\lambda\beta}) \\
    & = e^{-\lambda\alpha} - e^{-\lambda\beta}
\end{align*}

The properly normalised probability density function for $M\in[\alpha,\beta]$ is then
\[
    p_r(M;\boldsymbol{\theta}) = \frac{2f}{\erf\left(\frac{\beta - \mu}{\sigma}\right) - \erf\left(\frac{\alpha - \mu}{\sigma}\right)}s(M;\mu,\sigma) + \frac{1-f}{e^{-\lambda\alpha} - e^{-\lambda\beta}}b(M;\lambda)
\]
where $s(M;\mu,\sigma)$ and $b(M;\lambda)$ are as given on the sheet. Note that the value for $p_r(M)$ is zero when $M\notin[\alpha,\beta]$.
\section{Section B}


\end{document}